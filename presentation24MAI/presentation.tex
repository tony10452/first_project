

              
                \documentclass{beamer}
%
% Choose how your presentation looks.
%
% For more themes, color themes and font themes, see:
% http://deic.uab.es/~iblanes/beamer_gallery/index_by_theme.html
%
\mode<presentation>
{
  \usetheme{default}      % or try Darmstadt, Madrid, Warsaw, ...
  \usecolortheme{default} % or try albatross, beaver, crane, ...
  \usefonttheme{default}  % or try serif, structurebold, ...
  \setbeamertemplate{navigation symbols}{}
  \setbeamertemplate{caption}[numbered]
} 

\addtobeamertemplate{footline}{\hfill\insertframenumber/\inserttotalframenumber\hspace{2em}\null}

\usepackage[english]{babel}
\usepackage[utf8x]{inputenc}

\title[Your Short Title]{Présentation des travaux de thèse}
\author{Antoine GHORRA}
\institute{Département informatique et automatique- Institut Mines Telecom}
\date{24/05/2017}

\begin{document}

\begin{frame}
  \titlepage
\end{frame}

% Uncomment these lines for an automatically generated outline.
%\begin{frame}{Outline}
%  \tableofcontents
%\end{frame}

\section{Introduction}

\begin{frame}{Plan de la présentation}

\begin{itemize}
  \item Introduction
  \item Etat de l'art
  \begin{itemize}
  	\item Méthodologies de Clustering
  	\item Méthodologies de classification
  \end{itemize} 
  \item Présentation du projet d'article de review de l'état de l'art.
  \item Etudes et résultats sur DenStream
  \item Présentation de l'algorithme modifié
  \item Perspectives et travaux futurs.
\end{itemize}

\vskip 1cm


\end{frame}

\section{Some \LaTeX{} Examples}

\subsection{Tables and Figures}

\begin{frame}{Introduction}
	Suite aux travaux effectués au sein de l'équipe de recherche du département DIA viens le sujet de la thèse en question concernant le développement de méthodologies pour le classification/clustering des données de manière incrémentale et en ligne.

\begin{itemize}
	\item Pour effectuer une étude incrémentale et en ligne deux manières se présentent
\end{itemize}

\begin{table}
\centering
\begin{tabular}{l|r}
Classification & Clustering \\\hline
Apprentissage supervisé & Apprentissage non-supervisé \\
Notions de classes & Notions de Clusters
\end{tabular}
\caption{\label{tab:widgets}Quelques différences entre Classification et Clustering}
\end{table}


\end{frame}

\begin{frame}{Etat de l'art}


\begin{block}{Article Review Etat de l'art}
Après étude de l'état de l'art concernant les différents aspects de classification et/ou clustering qui peuvent être utilisés, on a songé à l'écriture d'un article de review de l'état de l'art qui sera présenté dans les deux semaines qui arrivent.
\end{block}
	
	\begin{block}{Base de données}
			La base de données utilisée dans les articles traitant les différentes méthodologies de clustering est KDD CUP 99'.
	\end{block}


\end{frame}

\begin{frame}{Etat de l'art}
	
	
	
\begin{table}[ht]
	\centering
	\resizebox{\textwidth}{!}{\begin{tabular}{|r|l|r|r|r|r|r|r|r|}
			\hline
			Facteur & DenStream & Clustream & D-Stream & Grid-based DBSCAN  \\ 
			\hline
			Flot continu & Oui & Oui & Oui & Oui  \\ \hline
			Nombre de Clusters connu & Non & Oui &  & \\ \hline
			Cluster de tailles aléatoire  & Oui & Non & Oui & Non \\ \hline
			Modèle à deux phases &  & Oui &  & Oui  \\ \hline 
			Traitement de données bruitées & Oui & Non & Oui & Oui \\	\hline
	\end{tabular}}
	\caption{Etude comparative des différentes méthodes de clustering en ligne incrémentale existante dans l'état de l'art} 
\end{table} 
	
	
\end{frame}

\begin{frame}{Expérimentations}
	DenStream est une méthodologie permettant de faire la détection des clusters dans un flot de données continue.
	
	DenStream est basé sur les concepts suivants:
	\begin{itemize}
		\item 
	\end{itemize}
	
	
\end{frame}

\subsection{Mathematics}

\begin{frame}{Readable Mathematics}

Let $X_1, X_2, \ldots, X_n$ be a sequence of independent and identically distributed random variables with $\text{E}[X_i] = \mu$ and $\text{Var}[X_i] = \sigma^2 < \infty$, and let
$$S_n = \frac{X_1 + X_2 + \cdots + X_n}{n}
      = \frac{1}{n}\sum_{i}^{n} X_i$$
denote their mean. Then as $n$ approaches infinity, the random variables $\sqrt{n}(S_n - \mu)$ converge in distribution to a normal $\mathcal{N}(0, \sigma^2)$.

\end{frame}

\end{document}
